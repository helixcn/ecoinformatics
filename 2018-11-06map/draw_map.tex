% Options for packages loaded elsewhere
\PassOptionsToPackage{unicode}{hyperref}
\PassOptionsToPackage{hyphens}{url}
%
\documentclass[
]{article}
\usepackage{amsmath,amssymb}
\usepackage{lmodern}
\usepackage{iftex}
\ifPDFTeX
  \usepackage[T1]{fontenc}
  \usepackage[utf8]{inputenc}
  \usepackage{textcomp} % provide euro and other symbols
\else % if luatex or xetex
  \usepackage{unicode-math}
  \defaultfontfeatures{Scale=MatchLowercase}
  \defaultfontfeatures[\rmfamily]{Ligatures=TeX,Scale=1}
\fi
% Use upquote if available, for straight quotes in verbatim environments
\IfFileExists{upquote.sty}{\usepackage{upquote}}{}
\IfFileExists{microtype.sty}{% use microtype if available
  \usepackage[]{microtype}
  \UseMicrotypeSet[protrusion]{basicmath} % disable protrusion for tt fonts
}{}
\makeatletter
\@ifundefined{KOMAClassName}{% if non-KOMA class
  \IfFileExists{parskip.sty}{%
    \usepackage{parskip}
  }{% else
    \setlength{\parindent}{0pt}
    \setlength{\parskip}{6pt plus 2pt minus 1pt}}
}{% if KOMA class
  \KOMAoptions{parskip=half}}
\makeatother
\usepackage{xcolor}
\usepackage[margin=1in]{geometry}
\usepackage{color}
\usepackage{fancyvrb}
\newcommand{\VerbBar}{|}
\newcommand{\VERB}{\Verb[commandchars=\\\{\}]}
\DefineVerbatimEnvironment{Highlighting}{Verbatim}{commandchars=\\\{\}}
% Add ',fontsize=\small' for more characters per line
\usepackage{framed}
\definecolor{shadecolor}{RGB}{248,248,248}
\newenvironment{Shaded}{\begin{snugshade}}{\end{snugshade}}
\newcommand{\AlertTok}[1]{\textcolor[rgb]{0.94,0.16,0.16}{#1}}
\newcommand{\AnnotationTok}[1]{\textcolor[rgb]{0.56,0.35,0.01}{\textbf{\textit{#1}}}}
\newcommand{\AttributeTok}[1]{\textcolor[rgb]{0.77,0.63,0.00}{#1}}
\newcommand{\BaseNTok}[1]{\textcolor[rgb]{0.00,0.00,0.81}{#1}}
\newcommand{\BuiltInTok}[1]{#1}
\newcommand{\CharTok}[1]{\textcolor[rgb]{0.31,0.60,0.02}{#1}}
\newcommand{\CommentTok}[1]{\textcolor[rgb]{0.56,0.35,0.01}{\textit{#1}}}
\newcommand{\CommentVarTok}[1]{\textcolor[rgb]{0.56,0.35,0.01}{\textbf{\textit{#1}}}}
\newcommand{\ConstantTok}[1]{\textcolor[rgb]{0.00,0.00,0.00}{#1}}
\newcommand{\ControlFlowTok}[1]{\textcolor[rgb]{0.13,0.29,0.53}{\textbf{#1}}}
\newcommand{\DataTypeTok}[1]{\textcolor[rgb]{0.13,0.29,0.53}{#1}}
\newcommand{\DecValTok}[1]{\textcolor[rgb]{0.00,0.00,0.81}{#1}}
\newcommand{\DocumentationTok}[1]{\textcolor[rgb]{0.56,0.35,0.01}{\textbf{\textit{#1}}}}
\newcommand{\ErrorTok}[1]{\textcolor[rgb]{0.64,0.00,0.00}{\textbf{#1}}}
\newcommand{\ExtensionTok}[1]{#1}
\newcommand{\FloatTok}[1]{\textcolor[rgb]{0.00,0.00,0.81}{#1}}
\newcommand{\FunctionTok}[1]{\textcolor[rgb]{0.00,0.00,0.00}{#1}}
\newcommand{\ImportTok}[1]{#1}
\newcommand{\InformationTok}[1]{\textcolor[rgb]{0.56,0.35,0.01}{\textbf{\textit{#1}}}}
\newcommand{\KeywordTok}[1]{\textcolor[rgb]{0.13,0.29,0.53}{\textbf{#1}}}
\newcommand{\NormalTok}[1]{#1}
\newcommand{\OperatorTok}[1]{\textcolor[rgb]{0.81,0.36,0.00}{\textbf{#1}}}
\newcommand{\OtherTok}[1]{\textcolor[rgb]{0.56,0.35,0.01}{#1}}
\newcommand{\PreprocessorTok}[1]{\textcolor[rgb]{0.56,0.35,0.01}{\textit{#1}}}
\newcommand{\RegionMarkerTok}[1]{#1}
\newcommand{\SpecialCharTok}[1]{\textcolor[rgb]{0.00,0.00,0.00}{#1}}
\newcommand{\SpecialStringTok}[1]{\textcolor[rgb]{0.31,0.60,0.02}{#1}}
\newcommand{\StringTok}[1]{\textcolor[rgb]{0.31,0.60,0.02}{#1}}
\newcommand{\VariableTok}[1]{\textcolor[rgb]{0.00,0.00,0.00}{#1}}
\newcommand{\VerbatimStringTok}[1]{\textcolor[rgb]{0.31,0.60,0.02}{#1}}
\newcommand{\WarningTok}[1]{\textcolor[rgb]{0.56,0.35,0.01}{\textbf{\textit{#1}}}}
\usepackage{graphicx}
\makeatletter
\def\maxwidth{\ifdim\Gin@nat@width>\linewidth\linewidth\else\Gin@nat@width\fi}
\def\maxheight{\ifdim\Gin@nat@height>\textheight\textheight\else\Gin@nat@height\fi}
\makeatother
% Scale images if necessary, so that they will not overflow the page
% margins by default, and it is still possible to overwrite the defaults
% using explicit options in \includegraphics[width, height, ...]{}
\setkeys{Gin}{width=\maxwidth,height=\maxheight,keepaspectratio}
% Set default figure placement to htbp
\makeatletter
\def\fps@figure{htbp}
\makeatother
\setlength{\emergencystretch}{3em} % prevent overfull lines
\providecommand{\tightlist}{%
  \setlength{\itemsep}{0pt}\setlength{\parskip}{0pt}}
\setcounter{secnumdepth}{-\maxdimen} % remove section numbering
\ifLuaTeX
  \usepackage{selnolig}  % disable illegal ligatures
\fi
\IfFileExists{bookmark.sty}{\usepackage{bookmark}}{\usepackage{hyperref}}
\IfFileExists{xurl.sty}{\usepackage{xurl}}{} % add URL line breaks if available
\urlstyle{same} % disable monospaced font for URLs
\hypersetup{
  pdftitle={draw\_map.R},
  pdfauthor={jlzhang},
  hidelinks,
  pdfcreator={LaTeX via pandoc}}

\title{draw\_map.R}
\author{jlzhang}
\date{2023-03-22}

\begin{document}
\maketitle

\begin{Shaded}
\begin{Highlighting}[]
\DocumentationTok{\#\#\#\# 导入所需要的程序包}
\FunctionTok{library}\NormalTok{(here)}
\end{Highlighting}
\end{Shaded}

\begin{verbatim}
## here() starts at C:/Users/jlzhang
\end{verbatim}

\begin{Shaded}
\begin{Highlighting}[]
\FunctionTok{library}\NormalTok{(rgdal)}
\end{Highlighting}
\end{Shaded}

\begin{verbatim}
## Loading required package: sp
\end{verbatim}

\begin{verbatim}
## Please note that rgdal will be retired during 2023,
## plan transition to sf/stars/terra functions using GDAL and PROJ
## at your earliest convenience.
## See https://r-spatial.org/r/2022/04/12/evolution.html and https://github.com/r-spatial/evolution
## rgdal: version: 1.6-5, (SVN revision 1199)
## Geospatial Data Abstraction Library extensions to R successfully loaded
## Loaded GDAL runtime: GDAL 3.5.2, released 2022/09/02
## Path to GDAL shared files: C:/Users/jlzhang/AppData/Local/R/win-library/4.2/rgdal/gdal
## GDAL binary built with GEOS: TRUE 
## Loaded PROJ runtime: Rel. 8.2.1, January 1st, 2022, [PJ_VERSION: 821]
## Path to PROJ shared files: C:/Users/jlzhang/AppData/Local/R/win-library/4.2/rgdal/proj
## PROJ CDN enabled: FALSE
## Linking to sp version:1.6-0
## To mute warnings of possible GDAL/OSR exportToProj4() degradation,
## use options("rgdal_show_exportToProj4_warnings"="none") before loading sp or rgdal.
\end{verbatim}

\begin{Shaded}
\begin{Highlighting}[]
\FunctionTok{library}\NormalTok{(tmap)}
\FunctionTok{library}\NormalTok{(tmaptools)}
\FunctionTok{library}\NormalTok{(sp)}
\FunctionTok{library}\NormalTok{(sf)}
\end{Highlighting}
\end{Shaded}

\begin{verbatim}
## Linking to GEOS 3.9.3, GDAL 3.5.2, PROJ 8.2.1; sf_use_s2() is TRUE
\end{verbatim}

\begin{Shaded}
\begin{Highlighting}[]
\FunctionTok{rm}\NormalTok{(}\AttributeTok{list =} \FunctionTok{ls}\NormalTok{())}

\DocumentationTok{\#\# 读取地图}

\CommentTok{\# 世界政区图shape文件}
\CommentTok{\# 来源: https://www.naturalearthdata.com/downloads/50m{-}cultural{-}vectors/}
\NormalTok{world }\OtherTok{\textless{}{-}} \FunctionTok{st\_read}\NormalTok{(}\StringTok{"ne\_50m\_admin\_0\_countries.shp"}\NormalTok{)}
\end{Highlighting}
\end{Shaded}

\begin{verbatim}
## Reading layer `ne_50m_admin_0_countries' from data source 
##   `C:\Users\jlzhang\Desktop\2018-11-06map\ne_50m_admin_0_countries.shp' 
##   using driver `ESRI Shapefile'
## Simple feature collection with 241 features and 94 fields
## Geometry type: MULTIPOLYGON
## Dimension:     XY
## Bounding box:  xmin: -180 ymin: -89.99893 xmax: 180 ymax: 83.59961
## Geodetic CRS:  WGS 84
\end{verbatim}

\begin{Shaded}
\begin{Highlighting}[]
\CommentTok{\# 国界shape文件}
\NormalTok{country }\OtherTok{\textless{}{-}} \FunctionTok{st\_read}\NormalTok{(}\StringTok{"bou1\_4l.shp"}\NormalTok{)}
\end{Highlighting}
\end{Shaded}

\begin{verbatim}
## Reading layer `bou1_4l' from data source 
##   `C:\Users\jlzhang\Desktop\2018-11-06map\bou1_4l.shp' using driver `ESRI Shapefile'
## Simple feature collection with 1382 features and 8 fields
## Geometry type: LINESTRING
## Dimension:     XY
## Bounding box:  xmin: 73.44696 ymin: 3.408477 xmax: 135.0858 ymax: 53.55793
## CRS:           NA
\end{verbatim}

\begin{Shaded}
\begin{Highlighting}[]
\CommentTok{\# 省界shape文件}
\NormalTok{province }\OtherTok{\textless{}{-}} \FunctionTok{st\_read}\NormalTok{(}\StringTok{"province\_polygon.shp"}\NormalTok{)}
\end{Highlighting}
\end{Shaded}

\begin{verbatim}
## Reading layer `province_polygon' from data source 
##   `C:\Users\jlzhang\Desktop\2018-11-06map\province_polygon.shp' 
##   using driver `ESRI Shapefile'
## Simple feature collection with 34 features and 10 fields
## Geometry type: MULTIPOLYGON
## Dimension:     XY
## Bounding box:  xmin: 73.44128 ymin: 18.15983 xmax: 135.0869 ymax: 53.56177
## Geodetic CRS:  Lat Long WGS84
\end{verbatim}

\begin{Shaded}
\begin{Highlighting}[]
\CommentTok{\# 设定投影为WGS84的经纬度投影}
\FunctionTok{st\_crs}\NormalTok{(world) }\OtherTok{\textless{}{-}} \StringTok{"EPSG:4326"}
\FunctionTok{st\_crs}\NormalTok{(country) }\OtherTok{\textless{}{-}} \StringTok{"EPSG:4326"}
\FunctionTok{st\_crs}\NormalTok{(province) }\OtherTok{\textless{}{-}} \StringTok{"EPSG:4326"}

\CommentTok{\# 城市数据}
\CommentTok{\# 读取世界城市人口数据 (假设为采样点),城市人口多少,用点的大小表示}
\NormalTok{city }\OtherTok{\textless{}{-}} \FunctionTok{read.csv}\NormalTok{(}\StringTok{"simplemaps{-}worldcities{-}basic.csv"}\NormalTok{, }\AttributeTok{header =} \ConstantTok{TRUE}\NormalTok{)}
\NormalTok{city }\OtherTok{\textless{}{-}}\NormalTok{ city[}\FunctionTok{sample}\NormalTok{(}\DecValTok{1}\SpecialCharTok{:}\FunctionTok{nrow}\NormalTok{(city), }\DecValTok{300}\NormalTok{), ] }\CommentTok{\# 随机筛选100个城市}

\FunctionTok{coordinates}\NormalTok{(city) }\OtherTok{\textless{}{-}} \ErrorTok{\textasciitilde{}}\NormalTok{ lng }\SpecialCharTok{+}\NormalTok{ lat}
\CommentTok{\# 转换为spatial dataframe,以便作为tmap的图层使用}

\CommentTok{\# 绘图}
\CommentTok{\# 只显示一部分,所以这里设定了xlim,ylim}
\FunctionTok{tm\_shape}\NormalTok{(world,}
         \AttributeTok{xlim =} \FunctionTok{c}\NormalTok{(}\DecValTok{60}\NormalTok{, }\DecValTok{140}\NormalTok{),}
         \AttributeTok{ylim =} \FunctionTok{c}\NormalTok{(}\DecValTok{0}\NormalTok{, }\DecValTok{60}\NormalTok{)) }\SpecialCharTok{+}
    \FunctionTok{tm\_borders}\NormalTok{(}\StringTok{"grey40"}\NormalTok{, }\AttributeTok{lwd =} \FloatTok{1.5}\NormalTok{) }\SpecialCharTok{+}
    \CommentTok{\# 先加载中国省级行政区(多边形 Polygon)}
    \FunctionTok{tm\_shape}\NormalTok{(province) }\SpecialCharTok{+}
    \FunctionTok{tm\_fill}\NormalTok{(}\AttributeTok{col =} \StringTok{"white"}\NormalTok{) }\SpecialCharTok{+}
    \FunctionTok{tm\_borders}\NormalTok{(}\StringTok{"grey60"}\NormalTok{,}
               \AttributeTok{lwd =} \FloatTok{0.8}\NormalTok{) }\SpecialCharTok{+}
    \CommentTok{\# 再加载国界线 (Polyline)}
    \FunctionTok{tm\_shape}\NormalTok{(country) }\SpecialCharTok{+}
    \FunctionTok{tm\_lines}\NormalTok{(}\AttributeTok{col =} \StringTok{"grey40"}\NormalTok{,}
             \AttributeTok{lwd =} \FloatTok{1.5}\NormalTok{) }\SpecialCharTok{+}
    \FunctionTok{tm\_scale\_bar}\NormalTok{(}\AttributeTok{position =} \FunctionTok{c}\NormalTok{(}\FloatTok{0.05}\NormalTok{, }\FloatTok{0.0}\NormalTok{)) }\SpecialCharTok{+}
    \FunctionTok{tm\_compass}\NormalTok{(}\AttributeTok{type =} \StringTok{"4star"}\NormalTok{,}
               \AttributeTok{position =} \FunctionTok{c}\NormalTok{(}\StringTok{"left"}\NormalTok{, }\StringTok{"top"}\NormalTok{)) }\SpecialCharTok{+}
    \FunctionTok{tm\_layout}\NormalTok{(}\AttributeTok{inner.margins =} \FunctionTok{c}\NormalTok{(}\FloatTok{0.12}\NormalTok{, }\FloatTok{0.03}\NormalTok{, }\FloatTok{0.08}\NormalTok{, }\FloatTok{0.03}\NormalTok{)) }\SpecialCharTok{+}
    \FunctionTok{tm\_shape}\NormalTok{(city) }\SpecialCharTok{+}
    \FunctionTok{tm\_bubbles}\NormalTok{(}\StringTok{"pop"}\NormalTok{,}
               \AttributeTok{col =} \StringTok{"red"}\NormalTok{,}
               \AttributeTok{scale =}\NormalTok{ .}\DecValTok{8}\NormalTok{,}
               \AttributeTok{border.col =} \StringTok{"red"}\NormalTok{) }\SpecialCharTok{+}
    \FunctionTok{tm\_legend}\NormalTok{(}\AttributeTok{legend.position =} \FunctionTok{c}\NormalTok{(}\FloatTok{0.05}\NormalTok{, }\FloatTok{0.08}\NormalTok{),}
              \AttributeTok{legend.stack =} \StringTok{"vertical"}\NormalTok{)}
\end{Highlighting}
\end{Shaded}

\begin{verbatim}
## Warning: Currect projection of shape city unknown. Long-lat (WGS84) is assumed.
\end{verbatim}

\begin{verbatim}
## Scale bar set for latitude km and will be different at the top and bottom of the map.
\end{verbatim}

\includegraphics{draw_map_files/figure-latex/unnamed-chunk-1-1.pdf}

\end{document}
